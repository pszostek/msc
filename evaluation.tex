\chapter{Evaluation}\label{chapter:evaluation}
Evaluation of the final system was performed in an automatic way using GROTOAP2 documents as input. We used a 5-fold cross-validation to better simulate behavior of the classifiers on unknown data and to rule out ``lucky'' division into subsets. In this way we will use all the samples for training and validation. In an $n$-fold cross-validation the data set is randomly partitioned into n subsets. In each of $n$ stages, a single subset is kept as the validation data and the remaining $n-1$ are used as training data. This process is repeated $n$ times and the results are summed up together.
To quantify goodness of the classification process for every class we will use precision and recall, which are two metrics complementing each other. Precision of classification is the fraction of correctly classified elements from a class to all samples/zones classified as this class. In other words, precision says what is the chance of a sample belonging to class A if it was classified as class A.

Recall of classification is the fraction of correctly classified elements from a class to all samples from the class. It intuitively indicates what fraction of a class was classified properly. 

\section{Initial classification evaluation} 
For the evaluation of the initial classification we used 262'144 samples. This is much less than available in GROTOAP2 (1'640'258). We needed to limit number of samples to keep the evaluation in a reasonable timescale. With the original number of samples it was not possible since:
\begin{itemize}
\item The training process involves matrix inversion, which has $\mathcal{O}(n^3)$ complexity with respect to the training set. This means that for the full documents set from GROTOAP2 it would take around 250 times more time.
\item \verb+libsvm+ is implemented using a single thread. It's therefore impossible to fully use a multi-core machine with this tool. 
\end{itemize}
Table \ref{tab:initial_confusion_matrix} presents results of a 5-fold cross-validation, including a confusion matrix, as well as precision and recall values.
\begin{table*}[]
\centering
\begin{tabular}{|r|c|c|c|c||c|c|}
\hline
& \rotatebox{90}{\textbf{BODY}} & \rotatebox{90}{\textbf{METADATA}} & \rotatebox{90}{\textbf{REFERENCES }} & \rotatebox{90}{\textbf{OTHER}} & \rotatebox{90}{\textbf{precision}} & \rotatebox{90}{\textbf{recall}} \\ \hline \hline
\textbf{BODY} & \textbf{167467} & 1394 & 350 & 769 & 96.95 & 98.52 \\ \hline
\textbf{METADATA} & 2094 & \textbf{51707} & 52 & 392 & 95.99 & 95.32  \\ \hline
\textbf{REFERENCES} & 1457 & 45 & \textbf{9641} & 97 & 95.67 & 87.77 \\ \hline
\textbf{OTHER} & 1724 & 719 & 34 & \textbf{24202}& 95.06 & 90.72 \\ \hline
\end{tabular}
\caption{Confusion matrix for the initial zone classification in CERMINE obtained in a 5-fold cross-validation. Rows and columns contain desired and obtained labels respectively.}
\label{tab:initial_confusion_matrix}
\end{table*}
\section{Metadata classification evaluation}
For the evaluation of the metadata classification we used 341'368 samples, therefore all available in GROTOAP2. Table \ref{tab:metadata_confusion_matrix} presents the confusion matrix together with precision and recall values.s

%%%%%%%%%%%%%%%%%%%%%%%%%%%%%%%%%%%%%%%%%%%%%%%%
\clearpage
\newgeometry{margin=1cm}
\thispagestyle{empty}
\begin{sidewaystable}[clockwise]
\centering
\begin{tabular}{|r||c|c|c|c|c|c|c|c|c|c|c|c||c|c|}
\hline \parbox[b]{\hsize}
& \rotatebox{65}{\textbf{abstract}} & \rotatebox{65}{\textbf{affiliation}} & \rotatebox{65}{\textbf{author}} & \rotatebox{65}{\textbf{bib\_info}} & \rotatebox{65}{\textbf{copyright}} & \rotatebox{65}{\textbf{correspondence}} & \rotatebox{65}{\textbf{dates}} & \rotatebox{65}{\textbf{editor}} & \rotatebox{65}{\textbf{keywords}} & \rotatebox{65}{\textbf{title}} & \rotatebox{65}{\textbf{title\_author}} & \rotatebox{65}{\textbf{type}} & \rotatebox{65}{\textbf{precision[\%]}} & \rotatebox{65}{\textbf{recall[\%]}} \\
\hline \hline
\textbf{abstract} & \textbf{30583} & 27 & 8 & 403 & 33 & 42  & 5  & 2 & 85 & 7 & 0 & 25 & 98.18 & 97.96 \\ \hline
\textbf{affiliation} & 40 & \textbf{17483} & 39 & 257 & 19 & 365 & 26 & 19 & 22 & 3 & 1 & 11 & 95.55 & 97.96 \\ \hline
\textbf{author}  & 12 & 59 & \textbf{12116} & 270 & 5 & 264 & 2 & 3 & 3 & 6 & 5 & 14 & 97.71 & 94.46 \\ \hline
\textbf{bib\_info} & 278 & 117 & 98 & \textbf{208099} & 117 & 496 & 317 & 2 & 133 & 66 & 10 & 359 & 97.62 & 99.05\\ \hline
\textbf{copyright} & 40 & 13 & 13 & 956 & \textbf{14419} & 58 & 40 & 0 & 5 & 1 & 14 & 25 & 95.72 & 92.58 \\ \hline
\textbf{correspondence} & 48 & 535 & 62 & 239 & 12 & \textbf{7783} & 15 & 1 & 12 & 1 & 0 & 0 & 90.04 & 89.38\\ \hline
\textbf{dates} & 13 & 5 & 2 & 1115 & 35 & 6 & \textbf{14026} & 1 & 11 & 4 & 1 & 0 & 97.15 & 92.16 \\ \hline
\textbf{editor} & 1 & 34 & 4 & 1 & 0 & 0 & 1 & \textbf{2264} & 0 & 0 & 0 & 0 & 98.69 & 98.22 \\ \hline
\textbf{keywords} & 104 & 23 & 21 & 880 & 1 & 9 & 3 & 2 & \textbf{4546} & 21 & 0 & 23 & 93.67 & 80.70 \\ \hline
\textbf{title} & 17 & 1 & 11 & 66 & 4 & 0 & 1 & 0 & 3 & \textbf{11642} & 25 & 21 & 98.61 & 98.74 \\ \hline
\textbf{title\_author} & 3 & 0 & 1 & 22 & 6 & 0 & 0 & 0 & 0 & 33 & \textbf{1565} & 0 & 96.53 & 96.00 \\ \hline
\textbf{type} & 12 & 0 & 25 & 872 & 33 & 0 & 2 & 0 & 33 & 22 & 0 & \textbf{7148} & 93.73 & 87.74 \\ \hline
\bottomrule
\end{tabular}
\caption{Confusion matrix obtained for classification of the metadata in CERMINE in a 5-fold cross-validation using documents from GROTOAP2. Rows and columns contain desired and obtained labels respectively.}
\label{tab:metadata_confusion_matrix}
\end{sidewaystable}

\clearpage
\restoregeometry
%%%%%%%%%%%%%%%%%%%%%%%%%%%%%%%%%%%%%%%%%%%%%%
\section{Future work}
\subsection{GROTOAP2}
GROTOAP2 is a rich data set created in a semi-automatized way from Open Access publications. It contains 13210 articles representing a big variety of layouts and issued by more than 200 publishers. We find this set very useful for every data mining algorithm willing to extract metadata, data and other various features from scholarly articles in an automated way. GROTOAP2 can be easily applied to testing or learning every algorithm designed for processing born-digital documents.
\quad
We believe that GROTOAP2 can be made more versatile by:
\begin{itemize}
\item growing its size by an order or two orders of magnitude.
\item Improving the algorithm used to assign NML metadata to text zones by tuning the magic constants used there.
\item Tuning the segmentation algorithm and improving segmentation quality. This would result in removing the \verb+author_title+ tag.
\item Dividing the \verb+body_content+ into subcategories based on the level of their headers.
\item Including fine-grained author metadata, e.g. first name, middle name, surname.
\item Introducing association between authors and theirs affiliations, i.e. pointing out which author is associated with which affiliation.
\item Including fine-grained bibliographic reference data, making it applicable for reference-extraction algorithms.
\end{itemize}
Most of the above-mentioned cases are related to loosing as little information as possible from the original Pubmed XML files.

\subsection{CERMINE}
CERMINE is a robust system for metadata extraction. It can be used to accurately extract fine-grained metadata from scientific articles en masse. CERMINE can be very useful in every digital library application, as it is flexible and doesn't depend on any particular layout. What is more, it can be easily adapted in a semi-automatic way to some particular layout, if such need arises.


We consider that there are many ways to improve the system, e.g.:
\begin{itemize}
\item increase classification accuracy and recall. This can be done by:
	\begin{itemize}
		\item choosing a different classification algorithm,
		\item using an ensemble of algorithms,
		\item performing lexical analysis on the zones classified as unknown,
		\item increasing quality of the training set.
	\end{itemize}
\item extract detailed bibliographic references,
\item extract figures and tables,
\item perform sentiment analysis with respect to the included references,
\item improve segmentation algorithm's on unusual layouts.
\end{itemize}