\chapter{Dataset}
\section{Motivation}
\subsection{Open Access publications}
\section{Previous work}
\section{Pubmed database}
\section{Dataset creation}
\subsection{The methodology of creating GROTOAP2}
\section{Metadata matching}

We use xpath to extract text entries from predefined paths. These
paths are common for the documents in the Pubmed dataset. Each path is
mapped to a label in our labeling system. The task consists in finding
a mapping betwen NLM entries and PDF zones, so that each zone could
obtain a label dependent on its path in the NLM file.

For each pair of kind (pdf\_zone, nlm\_entry) Smith-Waterman distance
is calculated (457-480)
For each zone a couple of different approaches are followed in order
to assign a correct label:
\begin{enumerate}
\item Take NLM entry with the highest ratio of SW distance to the
number of tokens (t1.alignment / t1.entryTokens.size()). If both PDF
zone and NLM entry are not empty and their size in tokens is
comparable (ratio not smaller than 0.7) and length of the common
substring constitutes (swLabelSim.get(zoneIdx).get(0).alignment /
entryTokens.size() > 0.7) more than 70% of the NLM entry, then assign
corresponding label  (483-516)
\item if the previous approach failed, take the entry with the highest
SW distance. If the length of the common substring is bigger than 50%
of the number of tokens in the PDF zone
(swLabelSim.get(zoneIdx).get(0).alignment / zoneTokens.size() > 0.5),
then assign corresponding label, (517-540)
\item if the previous aproaches failed, a \"cumulated\" distance is
calculated. This makes it possible to assign a label to these zones
that form together one NLM entry, but were segmented into several
parts. This applies mostly to BODY zones. For each and each entry and
each zone (SW\_distance / Math.max(zoneTokens.size(),
trio.entryTokens.size())) is calculated. For each zone these values
are aggregated by summing them up. From all the cumulated distance the
biggest one is taken. If it is greater than 0.5, the corresponding
label is assigned. (542-570)
\end{enumerate}
\section{Filtering}